\subsection{Rainfall climatology trends}

We investigated inter-annual trends in seasonal totals, storm depth and inter-storm arrival rate for the two rainy seasons. Overall, we find that changes in seasonal totals are minimal across the stations, shown in Table \ref{table:mk}, and as visualized for Jacobson Farm in Figure \ref{fig:jacobson}. However, we do find significant trends for increasing intensity, $\alpha$, and decreasing frequency, $\lambda$ in both rainfall seasons ($p < 0.05$, Table \ref{table:mk}) using the modified Mann-Kendall statistical test. Comparatively, total rainfall for both seasons and annual rainfall shows a muted change with fewer stations showing significant trends. Our results are consistent with those of \citeA{franz2010ecohydrological}, which analyzed a less recent dataset of 11 stations in the same region. In Figure \ref{fig:jacobson}, we use the Jacobson Farm rain gauge which has the longest record (79  years) to show the shifts in rainfall processes that occur in this region. Here, we see that while total rainfall for either season does not change significantly, we see an increase in $\alpha$ and decrease in $\lambda$ over the period ($p < 0.05$, Fig. \ref{fig:jacobson}). This indicates that storms are becoming more intense and less frequent. The relevance of this inter-annual variability in rainfall is further discussed in section \ref{discussion}.

\subsection{Non-stationarity impacts of seasonal water availability} \label{non-stationarity}

Figure \ref{fig:threefigs_dekadal} (bottom panel) shows the average soil moisture content for 10,000 growing seasons of a 180 day variety planted in dekad 7 (approx. March 1) using the Ol Jogi rainfall climatology. The $\alpha$ and $\lambda$ parameters used to generate rainfall are variable over the 6 month period. The inter-storm arrival rate, $\lambda$, increases more than three-fold during the first 80 days (8 dekads) of the season while the average depth per storm, $\alpha$, exhibits stationary variation between 8-13 mm of rain (Fig. \ref{fig:threefigs_dekadal}, top). On dekad 15, the $\lambda$, drops and levels off indicating the cessation of the long rains season and the  $\alpha$ value continues to oscillate. During this period of rainfall variability, the crop coefficient follows a step-wise function in which water requirements are low before steadily increasing when the crop begins to develop (Fig. \ref{fig:threefigs_dekadal}, middle). For the first 80 days of crop growth, the soil moisture levels increase steadily. At approximately day 80, the crop coefficient peaks at a value of 1.2, which is subsequently met with a decline in the soil water content (Fig. \ref{fig:threefigs_dekadal}, bottom). During this stage of peak water requirements from day 80 to 140, the crop enters its reproductive stage in which flowering and grain-filling occur. Concurrently, the water availability decreases as $\lambda$ values decline near the end of the long rains and the relative soil moisture, $s$ decreases from ca. 0.7 to 0.6. The variance of the relative soil moisture content fluctuates over the season, reaching its lowest point in the middle of the season between approximately days 82 and 100 (Fig. \ref{fig:threefigs_dekadal}, bottom). At the end of the growing season, the soil moisture levels slowly decline and flatten until day 180 when the crop coefficient linearly decrease to a final value of 0.6.

We then investigate the impact of stochastic rainfall on soil water availability for a single season. Figure \ref{fig:fig1}a shows an example time series of simulated daily rainfall for a late-maturing 180-day variety planted on day 60 (ca. March 1) using the Ol Jogi Farm rainfall climatology.  As we expect, volumetric water content increases in the middle of the season between days 50 and 90, which aligns with the peak of the rainfall season (Fig. \ref{fig:fig1}a and b).  The crop is moderately stressed over the entire season because the majority of the volumetric water content time series falls between the stress and wilting points. In this particular simulation, the crop experiences the lowest levels of stress  around day 75 whereas the highest levels of stress occur from the planting date until about 30 days later and for the last 70 days of the growing season. During these periods of high water stress, the water content time series crosses below the wilting point. %As rainfall events arrive, water content in the soil sharply increases. Later in the season around day 130, the water content rises above the stress point and the static stress is effectively is 0. 

For the simulated 10,000 growing seasons, we investigate the distributions and time series of average volumetric water content and static stress in Fig. \ref{fig:fig2}. We find a seasonality in water availability in which soil moisture content peaks between days (day of year) 135 and 145 for all simulations. The average soil water content begins to decrease around day 145 before stabilizing around day 170. The simulations are prone to water deficit during the earliest and latest parts of the season in which the average water content starts at 0.26 mm and ends the season around 0.28 mm, on average. The crop is at the highest points of stress during these periods as shown in Fig. \ref{fig:fig2}d. %since the stress point for a crop grown in these conditions is 0.37 mm. %could say something like: More than X% of simulations are exposed to Y events during Z part of the season.

Variance in volumetric water content and static stress is shown as confidence intervals for the 90th and 10th percentiles, which change after the first day of the season since all simulations start with the same initial value of $s$. The variability in results is much lower for volumetric water content than stress as shown by the smaller error bars in Fig. \ref{fig:fig2}b compared to \ref{fig:fig2}d. The crop is generally stressed over the course of the season because the stress point is 0.37 mm and the time series in Fig. \ref{fig:fig1}b falls below that for the majority of the season. The stress values of the crop ranges between 0 and 1, and in the time series, we see a larger increase in stress compared to the volumetric water content time series. Additionally, we see greater variability in stress values compared to volumetric water content values as the error bars for the 10th and 90th quantiles are much greater, especially in the later part of the season. This is due to the nonlinearity introduced in the conversion from relative soil moisture to stress.

\subsection{Dynamic Water Stress}
We have shown that changes in rainfall statistics alter the stress experienced by the crop. To convert average static stress into a yield metric that allows for the probability of crop failure, we use the dynamic stress equation. Dynamic water stress considers both the frequency and duration of excursions below the stress point and provides a yield estimate using a parsimonious boundary function. Calculating yield based on a parsimonious boundary function is typical in the literature on numerical simulations of climate variability and crop yields \cite{van2013yield, roche2020climate}.  

As shown in Fig. \ref{fig:dynstress}, the relationship between rainfall and yield is not a linear one, but rather exhibits a polynomial function with order less than 1. As the crop experiences more rainfall, yields increase up to a ceiling which is defined by the yield potential of the cultivar which in this case is a 180-day crop. We find that yields are closer to the maximum potential yield with seasonal rainfall totals greater than ca. 500 mm. When the seasonal rainfall totals are less than approximately 400 mm, the possibility of crop failure is introduced. We find that seasonal rainfall totals between 200 and 400 mm, there is a large range in possible outcomes for farmers. For a given seasonal total within this range, we find that farmers can reach up to 60 percent of the maximum possible yields while others experience total crop failure. For seasonal rainfall totals below 175 mm or so, all simulations result in total crop failure.

\subsection{Cultivar choice mediates probability of crop failure} \label{cultivar_choice}

We then investigate the effect of cultivar choice and rainfall climatology on yield. Figure \ref{fig:varietiesPDF} shows the joint probability distribution of yield and rainfall for three categories of maize varieties: early, medium and late maturing. The average rainfall and yield for each category of maize varieties is denoted as a black ``x". In Table \ref{varieties}, we provide the average statistics that correspond with Fig. \ref{fig:varietiesPDF}.

At lower levels of rainfall, early-maturing crops are more likely reach their maximum potential yields while medium and later maturing crops only gain a fraction of their yield as shown in Fig. \ref{fig:varietiesPDF}. However, under these low levels of rainfall, medium- and-late maturing crops also have a higher incidence of crop failure, which is indicated by the spread of probability densities around 0 yield between 0 and 200 mm of rainfall. In comparison, early-maturing crops are less likely to fail, which is indicated by the less dense probabilities below 200 mm of rainfall. These trends are reflected in the percentage of simulations that resulted in crop failure: approximately 18\%, 25\%, and 36\% for early-maturing, medium-maturing, and late-maturing, respectively, as shown in Table \ref{varieties}. Late-maturing varieties also have a larger spread in potential yields (top panel) compared to early- and medium-maturing varieties.

\subsection{Long-term trends in yield and crop failure}

After changing dekadal $\alpha$ and $\lambda$ values to represent those in an earlier era (1930s) and present day (2010s) at Jacobson Farm, we show how cropping outcomes have changed over the 80 year record in Fig. \ref{fig:cropoutcomesPDF} and Table \ref{summary_cropoutcomes}. We find changes in crop failure rates from the start of the record where the probability of crop failure was 38\% compared to present day crop failure at 34\%. The average crop failure in the middle of the record was 27\%. Similarly, crop yields have decreased from 0.95 t/ha at the start of the record to a present day value of 0.77 t/ha, with the largest yields in the middle of the record at 1.12 t/ha. 
The reasons for these differences stem from the shifting alpha and lambda values shown in Table \ref{summary_cropoutcomes} where the average alpha values increased from 5.71 mm per storm event in the 1930s to 10.99 mm per storm event in the 2010s. Conversely, the average lambda values decreased from 0.35 rainfall events per day in the 1930s to 0.16 rainfall events per day in the 2010s. As per the simulations, we see an increase in average rainfall between the 1930s and the middle of the record from 364 mm (standard deviation, SD, 138 mm) to 388 mm (SD 154 mm), and a decrease to the later part of the record in the 2010s of 316 mm (135 mm). The coefficient of variation of rainfall saw the highest value in the 2010s of 0.43 and the lowest in the 1930s, 0.38, with an average value in the middle of 0.40. 
