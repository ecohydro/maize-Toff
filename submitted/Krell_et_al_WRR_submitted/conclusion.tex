This stochastic ecohydrological model represents conditions that will become more common as climate variability and climate change alters rainfall in tropical and semiarid systems. By using historical rainfall to generate stochastic conditions for average depth and probability of rainfall we simulated a dryland environment for small-scale producers. We considered a common crop choice (maize), soil type, and planting decisions (i.e. timing of planting) that represents hundreds of millions of small-scale producers in regions vulnerable to climate change and variability. We investigated the role of rainfall variability in explaining current and past agricultural outcomes such as yield and likelihood of total crop failure. Additionally, we show the importance of cultivar choice in determining yield potentials and the vulnerability of late-maturing varieties to rainfall variability. 

How we characterized rainfall variability and its relationship to crop phenology (crop coefficient) was a novel contribution to a field where hydrologic processes are often considered separate from the phenology of the crop. We defined water availability as a function of stochastic rainfall and soil parameters whereas the crop coefficient governed water demand. Our model especially considers the rainfall depth and frequency (alpha and lambda) parameters as forces that drive stochastic rainfall. Thus we can simulate non-steady state soil moisture which changes over the course of the season due to shifts in the water required by the crop.

In the face of climate change, where direct changes in water availability, temperature, and increased prevalence of pest and diseases are possible, farmers can adapt through two key management strategies: choice of cultivar maturities and planting dates \cite{van2013yield}. Farmers need to select locally relevant planting dates and cultivars with appropriate maturities to minimize crop failure. Future study on the role of planting dates would illuminate the relative advantages of early- to late-maturing crops planted during Kenya's two primary rainy seasons. With appropriate climate data and locally-relevant agronomic conditions, this type of modeling can be used as a heuristic to improve our understanding of the impacts of climate variability on farming outcomes in other contexts. 